\documentclass{article}

\usepackage{amsmath, ../packages/enumitem, listings, graphicx, ../packages/jeffe}
\oddsidemargin 0in
\evensidemargin 0in
\textwidth 6.5in
\topmargin -0.5in
\textheight 9.0in

\begin{document}

\title{Multiple Source Shortest Path with unit weights}

% Title
\begin{center}
\textbf{\large Multiple Source Shortest Path with unit weights}
\end{center}

\section{Introduction}

\textbf{\underline{Given:}} Let $G$ be a directed graph $(V, \vec{E})$, embedded
on a surface with genus $g$. All edge weights are unit. \\
\textbf{\underline{Find:}} Consider boundary $f$ of $G$. $\forall v \in f$, find
a shortest path to $\forall u \in V$. \\

Let $T$ be the BFS (Breadth first search) tree of $G$, and $C$ be the BFS co-tree
in $G$. Then there is exactly $2g$ leftover edges $L = {e_1, e_2, \ldots, e_{2g}}$. \\

There exists a unique cycle $\lambda_i$ in $C \cup {e_i}$, and $(\lambda_1, 
\lambda_2, \ldots, \lambda_{2g}) = \Lambda$ defines homology basis. \\
We define homological signature of an edge as follows:
\[ [e]_{i} = \begin{cases} 1 & ,\mbox{if } e \in \lambda_i \\
                          -1 & ,\mbox{if } rev(r) \in \lambda_i \\
                           0 & ,\mbox{otherwise} \end{cases}\]

Furthermore, we define $\alpha$ recursively as follows: \\
\[ \alpha(\vec{e}^*) = \begin{cases} 1 & ,\mbox{if } e^* \mbox{ is a leaf edge of } C \\
                           \sum \limits_{ \text{tail}(\vec{e}^{'*})
                           = \text{head}(\vec{e}^*) } \alpha(\vec{e}^{'}) & ,
                           \mbox{otherwise} \end{cases}\]
We can extend above definition with $\alpha(\vec{e}) = \alpha( \vec{e}^* )$ and 
  $\alpha(e)^* = - \alpha(\text{rev}(\vec{e}^*))$. \\

Let $\tilde w(\vec{e}) = \langle 1, [\vec{e}], \alpha(\vec{e}) \rangle$ be new
weight vector for each edge in $G$.
\begin{center}
\includegraphics[scale = 0.3]{alphaDef.jpg}
\end{center}

\textbf{\underline{Def:}} An edge $\vec{e}$ is "holier" than $\vec{e}^{'}$, 
if $\tilde w(\vec{e}) < \tilde w(\vec{e}^{'})$ in lexicographic comparison. 
Therefore, we can define "holiness" of any $S \subset G $ as follows:
\[\text{Ho}(S) = \sum \limits_{\vec{e} \in S} \tilde w(\vec{e})\]



\newpage

Holiest tree is a spanning tree with minimal "holiness". We build Holiest tree 
rooted at $r$, using slight tweak in the Bellman-Ford algorithm for finding 
shortest path tree rooted at r. \\

\begin{minipage}[t]{0.48\linewidth}
\begin{algorithm}
\textbf{\underline{BuildHoliestTree}}($G, \tilde w , r$): \\ \quad
Set $d[r] \leftarrow \langle 0, [0], 0 \rangle$ \\ \quad \quad
    pred($r$) $\leftarrow$ NULL \\ \quad
for all $v : v \neq r$ \\ \quad \quad
    $d[r] \leftarrow \langle \infty, [\infty], \infty \rangle$ \\ \quad \quad
    pred($r$) $\leftarrow$ NULL \\ \quad
put $r$ into queue \\ \quad
while queue is not Empty: \\ \quad \quad
    Let $s \leftarrow$ dequeue item \\ \quad \quad
    for all $u: s \rightarrow u$ \\ \qquad \quad
        if $u$ is not marked \\ \quad \qquad \quad
           mark $u$ and put in the queue \\ \qquad \quad
        if isTense($s \rightarrow u$) \\ \quad \qquad \quad
           relax($s \rightarrow u$)
\end{algorithm}
\end{minipage}
\hfill%
\hspace{-4cm}
\begin{minipage}[t]{0.48\linewidth}
\begin{algorithm}
\textbf{\underline{isTense}}($s \rightarrow u$): \\ \quad
return $d[s] + \tilde w(s \rightarrow u) < d[u]$ \\

\end{algorithm}

\vspace{0.5cm}

\begin{algorithm}
\textbf{\underline{relax}}($s \rightarrow u$): \\ \quad
$d[u] \leftarrow d[s] + \tilde w(s \rightarrow u)$ \\ \quad
pred[u] $\leftarrow s$ \\
\end{algorithm}
\end{minipage}

\vspace{0.5cm}

\textbf{\underline{Observation:}} Each edge will be added once to the queue. \\

\textbf{\underline{Corollary:}} Each edge will be relaxed at most once. \\

\textbf{\underline{Lemma-1:}} If there is no tense edge in $G$, then for each 
$v: r \rightarrow \ldots \rightarrow \text{pred}(\text{pred}(v)) \rightarrow 
\text{pred}(v) \rightarrow v$ is the holiest path from $r$ to $v$.

\begin{proof} Let's prove it by induction on $d[v]$ distance from the root $r$. \\
\underline{Base:} $d[v] = 0$, then v = r, so the claim is definitely true. \\
\underline{Induction Step:} Suppose the claim is true for all vertex $v \in V$ 
such that $d[v] < d$ for some d. Consider vertex v such that $d[v] = d$. 
By induction hypothesis, all vertices with $d[u] = d-1$ have "holiest" path 
correctly updated. By definition, $d[v] = \text{min}_{s \rightarrow v}{d[s] + 
\tilde w(s \rightarrow v)}$, here $d[s] = d-1$. By Induction hypothesis, $d[s]$ 
is "holiest" path, so if there is no tense edge in G then 
$d[v] = \text{min}_{s \rightarrow v}{d[s] + \tilde w(s \rightarrow v)}$ holds.
\end{proof}

\textbf{\underline{Corollary:}} The algorithm will produce "holiest" tree rooted at r in linear time.

We now have produced our initial "Holiest" tree.

\newpage
\section{Working on examples:}
\textbf{Holiest Tree:} \\
On genus $g = 1$ surface:
\begin{center}
\includegraphics[scale = 0.6]{g1.jpg}
\end{center}

\newpage
On genus $g = 3$ surface:
\begin{center}
\includegraphics[scale = 0.6]{g3.jpg}
\end{center}

\newpage
\section{References:}
\begin{itemize}
\item Cabello, Sergio, Erin W. Chambers, and Jeff Erickson. "Multiple-source 
  shortest paths in embedded graphs." SIAM Journal on Computing 42.4 (2013): 1542-1571.
\item Eisenstat, David, and Philip N. Klein. "Linear-time algorithms for max 
  flow and multiple-source shortest paths in unit-weight planar graphs." 
  Proceedings of the forty-fifth annual ACM symposium on Theory of computing. ACM, 2013.
\item Erickson, Jeff. "Maximum flows and parametric shortest paths in planar graphs." 
  Proceedings of the twenty-first annual ACM-SIAM symposium on Discrete Algorithms. 
  Society for Industrial and Applied Mathematics, 2010.
\end{itemize}
\end{document}
