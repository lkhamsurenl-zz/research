\documentclass{article}

\usepackage{amsmath, ../packages/enumitem, listings, graphicx, ../packages/jeffe}
\oddsidemargin 0in
\evensidemargin 0in
\textwidth 6.5in
\topmargin -0.5in
\textheight 9.0in

\begin{document}

\title{Multiple Source Shortest Path with unit weights}

% Title
\begin{center}
\textbf{\large Multiple Source Shortest Path with unit weights}
\end{center}

\section{Introduction}

\textbf{\underline{Given:}} Let $G$ be a directed graph $(V, \vec{E})$, embedded
on a surface with genus $g$. All edge weights are unit. \\
\textbf{\underline{Find:}} Consider boundary $f$ of $G$. $\forall v \in f$, find
a shortest path to $\forall u \in V$. \\

Let $T$ be the BFS (Breadth first search) tree of $G$, and $C$ be the BFS co-tree
in $G$. Then there is exactly $2g$ leftover edges $L = \{e_1, e_2, \ldots, e_{2g}\}$. \\

There exists a unique cycle $\lambda_i$ in $C \cup {e_i}$, and $(\lambda_1, 
\lambda_2, \ldots, \lambda_{2g}) = \Lambda$ defining homology basis. \\
We define homological signature of an edge as follows:
\[ [e]_{i} = \begin{cases} 1 & ,\mbox{if } e \in \lambda_i \\
                          -1 & ,\mbox{if } rev(r) \in \lambda_i \\
                           0 & ,\mbox{otherwise} \end{cases}\]

Furthermore, we define leafmost term $\alpha$ recursively as follows: \\
\[ \alpha(\vec{e}^*) = \begin{cases} 1 & ,\mbox{if } e^* \mbox{ is a leaf edge of } C \\
                           \sum \limits_{ \text{tail}(\vec{e}^{'*})
                           = \text{head}(\vec{e}^*) } \alpha(\vec{e}^{'}) & ,
                           \mbox{otherwise} \end{cases}\]
We can extend above definition with $\alpha(\vec{e}) = \alpha( \vec{e}^* )$ and 
  $\alpha(e)^* = - \alpha(\text{rev}(\vec{e}^*))$. \\

Let $\tilde w(\vec{e}) = \langle 1, [\vec{e}], \alpha(\vec{e}) \rangle$ be new
weight vector for each edge in $G$.
\begin{center}
\includegraphics[scale = 0.3]{alphaDef.jpg}
\end{center}

\textbf{\underline{Def:}} An edge $\vec{e}$ is "holier" than $\vec{e}^{'}$, 
if $\tilde w(\vec{e}) < \tilde w(\vec{e}^{'})$ in lexicographic comparison. 
Therefore, we can define "holiness" of any $S \subset G $ as follows:
\[\text{Ho}(S) = \sum \limits_{\vec{e} \in S} \tilde w(\vec{e})\]



\newpage

Holiest tree is a spanning tree with minimal "holiness". We build Holiest tree 
rooted at $r$, using slight tweak in the Bellman-Ford algorithm for finding 
shortest path tree rooted at r. \\

\begin{minipage}[t]{0.48\linewidth}
\begin{algorithm}
\textbf{\underline{BuildHoliestTree}}($G, \tilde w , r$): \\ \quad
Set $d[r] \leftarrow \langle 0, [0], 0 \rangle$ \\ \quad \quad
    pred($r$) $\leftarrow$ NULL \\ \quad
for all $v : v \neq r$ \\ \quad \quad
    $d[r] \leftarrow \langle \infty, [\infty], \infty \rangle$ \\ \quad \quad
    pred($r$) $\leftarrow$ NULL \\ \quad
put $r$ into queue \\ \quad
while queue is not empty: \\ \quad \quad
    Let $u \leftarrow$ dequeue item \\ \quad \quad
    for all $u \rightarrow v$ \\ \qquad \quad
        if $v$ is not marked \\ \quad \qquad \quad
           mark $v$ and put in the queue \\ \qquad \quad
        if isTense($u \rightarrow v$) \\ \quad \qquad \quad
           relax($u \rightarrow v$)
\end{algorithm}
\end{minipage}
\hfill%
\hspace{-4cm}
\begin{minipage}[t]{0.48\linewidth}
\begin{algorithm}
\textbf{\underline{isTense}}($u \rightarrow v$): \\ \quad
return $d[u] + \tilde w(u \rightarrow v) < d[v]$ \\

\end{algorithm}

\vspace{0.5cm}

\begin{algorithm}
\textbf{\underline{relax}}($u \rightarrow v$): \\ \quad
$d[v] \leftarrow d[u] + \tilde w(u \rightarrow v)$ \\ \quad
pred[v] $\leftarrow u$ \\
\end{algorithm}
\end{minipage}

\vspace{0.5cm}

\textbf{\underline{Observation:}} Each vertex will be added once to the queue. \\

\textbf{\underline{Corollary:}} Each edge will be relaxed at most once. \\

\textbf{\underline{Lemma-1:}} If there is no tense edge in $G$, then for each 
$v: r \rightarrow \ldots \rightarrow \text{pred}(\text{pred}(v)) \rightarrow 
\text{pred}(v) \rightarrow v$ is the holiest path from $r$ to $v$.

\begin{proof} Let's prove it by induction on $d[v][0]$ distance from the root $r$. \\
\underline{Base:} $d[v][0] = 0$, then v = r, so the claim holds trivially. \\
\underline{Induction Step:} Suppose the claim is true for all vertex $v \in V$ 
such that $d[v][0] < d$ for some d. Consider vertex v such that $d[v][0] = d$. 
By induction hypothesis, all vertices with $d[u][0] = d-1$ have "holiest" path 
correctly updated. By definition, $d[v] = \text{min}_{u \rightarrow v}{d[u] + 
\tilde w(u \rightarrow v)}$, here $d[u][0] = d-1$. By Induction hypothesis, $d[u]$ 
is not tense and can construct "holiest" path to $u$, so if there is no tense
edge in G then $d[v] = \text{min}_{u \rightarrow v}{d[u] + \tilde w(u \rightarrow v)}$ 
holds.
\end{proof}

\textbf{\underline{Corollary:}} The algorithm will produce "holiest" tree rooted at r in linear time.

We now have produced our initial "Holiest" tree.

\section{Moving Along an Edge}
Consider a single edge uv, which is on the boundary face $f$ of G. Suppose we
already computed the holy-tree $T_u$ rooted at u.  We transform $T_u$
into the holy-tree $T_v$ as follows. First, we insert a new vertex s in the
interior of the uv, bisecting it into two edges su and sv with weights:
\[ w_{0}(s \rightarrow u) = \langle 0, [\vec{0}], 0 \rangle \]
\[ w_{0}(s \rightarrow v) = \langle 1, [w(u \rightarrow v)], 
  \alpha(w(u \rightarrow v)) \rangle = w(u \rightarrow v)\]

Observe that this condition implies $s = u$, therefore $T_s = T_u$. We reduce 
distances to u and v as follows:

\[ w_{\epsilon}(s \rightarrow u) = \langle 0, -[w(u \rightarrow v)], 
  -\alpha(w(u \rightarrow v)) \rangle \]
\[ w_{\epsilon}(s \rightarrow v) = \langle 1, [\vec{0}], 0 \rangle \]

Since we reduced distance to all vertices in the graph equally, the process does
not introduce any pivots. Then we define a parametric weights as follows:

\[ w_{\lambda}(s \rightarrow u) = \langle 0, -[w(u \rightarrow v)], 
  -\alpha(w(u \rightarrow v)) \rangle - \lambda \]
\[ w_{\lambda}(s \rightarrow v) =  \langle 1, [\vec{0}], 0 \rangle  + \lambda \]

Every other dart $x \rightarrow y$ has constant parametric weight 
$w_{\lambda}(x \rightarrow y) = w(x \rightarrow y)$.
We then maintain the holy tree $T_{\lambda}$ rooted at s, with respect
to the weight function $w_{\lambda}$, as $\lambda$ increases continuously from
0 to $\langle 1, [\vec{0}], 0 \rangle$. When $\lambda = w(u \rightarrow v)$, 
$T_{\lambda} = T_v$.

In the following algorithm, \textbf{pred} defines Holy tree rooted at $u$, and 
\textbf{dist} is corresponding distance to each vertex in the graph. \\

% To ensure that introducing $dist_0[s \rightarrow u] = \langle 0, -2 * [
% w(u \rightarrow v)], -2 * \alpha(w(u \rightarrow v)) \rangle$ does 
% not introduce any dart with negative slack, we carefully choose fundamental 
% homology cycles to not cross any of the darts of boundary face f. If we assume
% that moving around the face process is counter-clockwise as shown below,
% $\forall u_i -> u_{i + 1} \in f, \alpha(w(u_i \rightarrow u_{i + 1})) <= 0$.
% Therefore, $dist[s \rightarrow u] = \langle 0, -2 * [
% w(u \rightarrow v)], -2 * \alpha(w(u \rightarrow v)) \rangle  + \lambda
% = \langle 0, [0], -2 * \alpha(w(u \rightarrow v)) \rangle  + \lambda$. Since we
% are adding non-negative distance to u, no slack will be nonnegative by
% construction. 

\begin{algorithm}
\textbf{\underline{MoveAlongEdge}}($G, u \rightarrow v, dist, pred$): \\ \quad
Add new vertex s \\ \quad
$pred[u], pred[v] \leftarrow s$ \\ \quad
$\lambda \leftarrow 0$ \\ \\ \quad

$w(s \rightarrow u) \leftarrow \langle 0, -[w(s \rightarrow u)], 
-\alpha(w(s \rightarrow u)) \rangle$ \\ \quad
AddSubtree$(\langle 0, -[w(s \rightarrow u)], 
-\alpha(w(s \rightarrow u)) \rangle, u)$ \\ \\ \quad

$w(s \rightarrow v) \leftarrow \langle 1, [\vec{0}], 0 \rangle$ \\ \quad
AddSubtree$(\langle 0, -[w(s \rightarrow u)], 
-\alpha(w(s \rightarrow u)) \rangle, v)$ \\ \\ \quad

while $\lambda < \langle 1, [\vec{0}], 0 \rangle$: \\ \quad \quad
    \textbf{pivot} $\leftarrow $ FindNextPivot \\ \quad \quad
    If \textbf{pivot} is non NULL \textbf{AND} 
    $(\lambda + slack(\textbf{pivot}) / 2) < \langle 1, [\vec{0}], 0 \rangle$ \\ \qquad \quad
        Pivot(\textbf{pivot}) \\ \qquad \quad
        $\lambda \leftarrow \lambda + slack(\textbf{pivot}) / 2$ \\ \quad \quad
    else \\ \qquad \quad
        $\delta = \langle 1, [\vec{0}], 0 \rangle - \lambda$ \\ \qquad \quad
        AddSubtree$(\delta, u)$ \\ \qquad \quad
        AddSubtree$(-\delta, v)$ \\ \qquad \quad
        $\lambda \leftarrow \lambda + \delta$ \\
\end{algorithm}

\newpage
\section{Working on examples:}
\textbf{Holiest Tree:} \\
On genus $g = 1$ surface:
\begin{center}
\includegraphics[scale = 0.6]{g1.jpg}
\end{center}

\newpage
On genus $g = 3$ surface:
\begin{center}
\includegraphics[scale = 0.6]{g3.jpg}
\end{center}

\newpage
\section{References:}
\begin{itemize}
\item Cabello, Sergio, Erin W. Chambers, and Jeff Erickson. "Multiple-source 
  shortest paths in embedded graphs." SIAM Journal on Computing 42.4 (2013): 1542-1571.
\item Eisenstat, David, and Philip N. Klein. "Linear-time algorithms for max 
  flow and multiple-source shortest paths in unit-weight planar graphs." 
  Proceedings of the forty-fifth annual ACM symposium on Theory of computing. ACM, 2013.
\item Erickson, Jeff. "Maximum flows and parametric shortest paths in planar graphs." 
  Proceedings of the twenty-first annual ACM-SIAM symposium on Discrete Algorithms. 
  Society for Industrial and Applied Mathematics, 2010.
\end{itemize}
\end{document}
