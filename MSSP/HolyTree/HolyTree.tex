\documentclass{article}

\usepackage{jeffe, enumitem}
\usepackage{amsmath,listings, graphicx, caption, subcaption, float, amssymb, hyperref}

\usepackage[T2A,T1]{fontenc}
\DeclareSymbolFont{cyrillic}{T2A}{cmr}{m}{n}
\DeclareMathSymbol{\Sha}{\mathalpha}{cyrillic}{216}

\usepackage[
backend=biber,
style=numeric,
sorting=nyt
]{biblatex}
\addbibresource{citation.bib}

\oddsidemargin 0in
\evensidemargin 0in
\textwidth 6.5in
\topmargin -0.5in
\textheight 9.0in

\definecolor{electricpurple}{rgb}{0.75, 0.0, 1.0}
\begin{document}

\title{Multiple Source Shortest Path with unit weights}

% Title
\begin{center}
\textbf{\large Multiple-source shortest paths with unit weights in embedded
graphs}
\end{center}

\DRAFT

\begin{bigabstract}
We describe a new algorithm that computes multiple-source shortest paths from
vertices in a given face to all other vertices in an embedded graph
with unit weight edges. Furthermore, algorithms implementation and visualization can be found \href{https://github.com/lkhamsurenl/research/tree/master/HolyImpl/src}{on Github}\footnote{\url{https://github.com/lkhamsurenl/research/tree/master/HolyImpl/src}}.
\end{bigabstract}

\section{Introduction}

We can formally define multiple-source shortest paths problem as follows: 

\begin{oneshot}{Given}
Let $G$ be a directed graph $(V, \vec{E})$ with unit edge weights, embedded on a surface with genus $g$.
\end{oneshot}

\begin{oneshot}{Find}
Consider an arbitrary face $f$ of $G$. $\forall v \in f$, find a shortest path to
$\forall u \in V$.
\end{oneshot}

Klein \cite{klein2005multiple} described first algorithm to solve MSSP problem in planar graphs in $O(n \log{n})$-time. Later, Cabello, Chamber, and Erickson \cite{cabello2013multiple} described an alternate method and generalized it to graphs embedded in higher genus surfaces. 
Recently, Eisenstat and Klein \cite{eisenstat2013linear} introduced a new algorithm to compute MSSP in planar graphs in linear time when edge weights are unit. Our 
paper attempts to generalize this idea to graphs embedded in higher genus surfaces. In their paper, Eisenstat and Klein use so-called leafmost pivoting rule to deal with possibility more than one pivots at the given time. However, there is no direct way 
to generalize a leafmost pivoting rule in graphs embedded in higher genus surfaces 
In this paper, we describe how to alleviate this difficulties and how to compute MSSP. 

\subsection{Definition and Notation}
We adapt definitions and notations for surfaces, embedding, and duality from Cabello et al.'s paper \cite{cabello2013multiple}. Furthermore, we refer to graphs embedded on higher genus surfaces as ``embedded graphs''.

\section{Holiest Tree}
Let $T$ be a breadth first search(BFS) tree of $G$, and $C^{*}$ be a BFS 
co-tree in $(G/T)^*$. Then there is exactly $2g$ leftover edges 
$L = \{e_1, e_2, \ldots, e_{2g}\}$, according to the Euler's formula. \\

There exists a unique cycle $\lambda_i$ in $C^{*} \cup {e_i}$, and $(\lambda_1, 
\lambda_2, \ldots, \lambda_{2g}) = \Lambda$ defining homology basis. \\
We define homological signature of an edge as follows:
\[ [e]_{i} = \begin{cases} 1 & ,\mbox{if } e \in \lambda_i \\
                          -1 & ,\mbox{if } rev(e) \in \lambda_i \\
                           0 & ,\mbox{o/w} \end{cases}\]

Furthermore, we define $\alpha$  term recursively for $C^*$ as follows: \\
\[ \alpha(e^*) = 
  \begin{cases} 1 & ,\mbox{if } rev(e^*) \mbox{ is a leaf dart in } C^{*} \\
                           \sum \limits_{ \text{tail}(e^{'*})
                           = \text{head}(e^*) } \alpha(e^{'}) & ,
                           \mbox{o/w} \end{cases}\]
We can extend above definition with $\alpha(e) = \alpha(e^*)$ and 
  $\alpha(e)^* = - \alpha(\text{rev}(e^*))$. \\

This $\alpha$ term definition was first introduced by Cunningham \cite{cunningham1976network}.

\begin{oneshot}{Observation}
For any cycle C in G, $\abs{\alpha(C)} = \text{number of faces in C}$.
\end{oneshot}

Let $\tilde w(e) = ( 1, \vec{[e]}, \alpha(e) )$ be new
weight vector for each edge in $G$. We refer to each component this weight vector
as length, homology, and $\alpha$ terms respectively. \\

We extend above definition of holiness to set of edges $S$ as follows:
\[\text{H}(S) = \sum \limits_{e \in S} \tilde w(e)\]
An edge $e_1$ is holier than $e_2$, 
if $\tilde w(e_1) < \tilde w(e_2)$ in lexicographic comparison. 
Similarly, we say $S_1$ is holier than $S_2$, if $H(S_1) < H(S_2)$. \\

Holiest tree of G is a spanning tree with minimal holiness. Note here that computing holiest tree rooted at $s$ immediately gives us singe
source shortest path tree rooted at $s$ in our case. \\

We show an example figure for holiest tree computation for a grid graph 
embedded in genus 1 surface. Notice that boundary edges are replicated for nice 
visual. Bold thick arrows represent homology cycles, grey arrows correspond to 
$\alpha$ terms. Holiest tree is rooted at $(1, 1)$. 

\begin{figure}[H]
  \label{fig: g1}
  \centering
  \includegraphics[scale = 0.55]{figures/g1.jpg}
  \vspace{-3.5cm} % Remove the space after the picture.
  \caption{An example of holiest tree in graph with genus 1.}
\end{figure}

\subsection{Need of holiness}

\begin{oneshot}{Lemma 2.1}
Let $p_1$ and $p_2$ be any two distinct paths between vertices $x$ and $y$ in an embedded graph G. Then $H(p_1) \neq H(p_2)$ holds. 
\end{oneshot}
\begin{proof}
Suppose $H(p_1) = H(p_2)$ holds. Let $\gamma$ be a cycle formed by concatanating $p_1$ and $rev(p_2)$.
\begin{align}
& H(p_1) = H(p_2) \Rightarrow H(p_1 + rev(p_1)) = 0 \Rightarrow  H(\gamma) = 0 \Rightarrow \\
&  \Rightarrow [\gamma] = 0 \text{ and } \alpha(\gamma) = 0.
\end{align}

$[\gamma] = 0$ implies that $\gamma$ is null-homologous, therefore a separating cycle in $\Sigma$. Suppose we get two separate connected components $\Sigma_1$ and $\Sigma_2$, which includes face $f$, if we split $\Sigma$ with $\gamma$. \\

\begin{figure}[H]
  \label{fig: g2_separating_cycle}
  \centering
  \includegraphics[scale = 0.55]{figures/g2_separating_cycle.png}
  \caption{Separating cycle $\gamma$.}
\end{figure}

According to the definition of $\alpha$ term, $\abs{\alpha(\gamma)}$ is exactly equal to number of faces in $\Sigma_1$. However, there is at least one face in $\Sigma_1$, so $\abs{\alpha(\gamma)} > 0$, which contradicts to $\abs{\alpha(\gamma)} = 0$ in $(2)$.
\end{proof}

\begin{oneshot}{Corollary 2.1}
$\forall v \in f$, there is a unique holiest path from $v$ to $\forall u \in V$.
\end{oneshot}

\subsubsection{Leftmost tree vs Holiest tree}

Klein \cite{klein2005multiple} used a notion of leftmost tree to resolve ambiguity when 
computing MSSP in their paper.
Our initial hope was that even in higher genus surfaces leftmost tree and holiest
tree would be equal. However, on genus $g = 4$ surface, we show that no
choice of ordering and directions of homology and $\alpha$ terms can result in equal holiest and
leftmost trees. Consider below picture, \\
\begin{itemize}
\item Vertices and edges in boundaries are replicated twice, except vertex $J$, which occurs 16 times in total.
\item Leftmost tree is rooted at $A$ and denoted by solid \textbf{black} darts.
\item {\color{red} Co-tree} is rooted at face enclosed by vertices $A, B, F, E$. Darts of the tree are denoted in red.
\item There are exactly 8 homology cycles, each specifically denoted with
special darts crossing the boundaries. Those edges make up $L = \{e_1, e_2, \ldots, e_{2g}\}$, leftover edges.
\end{itemize}
We observe that 2 darts highlighted in blue will be tense, 
regardless of the homology and $\alpha$ choice configurations.
\begin{oneshot}{Observation}
Regardless of the choice of directions for all homology cycles, $g \rightarrow \ltimes$ or 
$\Sha \rightarrow \square$ will be tense (highlighted in {\color{blue} blue} in the picture). 
\end{oneshot}

\begin{center}
\includegraphics[angle = -90, scale = 0.7]{figures/Genus4_InitialHolyTree.jpg}
\end{center}

\subsection{Initial holiest tree}
Holiest tree is a spanning tree with minimal holiness. We build holiest tree 
rooted at $r$, using slight tweak in the Bellman-Ford algorithm for finding 
shortest path tree rooted at r. \\

\begin{minipage}[t]{0.58\linewidth}
\begin{algorithm}
\textbf{\underline{BuildHoliestTree}}($G, \tilde w , r$): \\ \quad
Set $dist[r] \leftarrow ( 0, \vec{[0]}, 0 )$ \\ \quad \quad
    pred($r$) $\leftarrow$ NULL \\ \quad
for all $v : v \neq r$ \\ \quad \quad
    $dist[r] \leftarrow ( \infty, \vec{[\infty]}, \infty )$ \\ \quad \quad
    pred($r$) $\leftarrow$ NULL \\ \quad
put $r$ into queue \\ \quad
while queue is not empty: \\ \quad \quad
    Let $u \leftarrow$ dequeue item \\ \quad \quad
    for all $u \rightarrow v$ \\ \qquad \quad
        if $v$ is not marked \\ \quad \qquad \quad
           mark $v$ and put in the queue \\ \qquad \quad
        if isTense($u \rightarrow v$) \\ \quad \qquad \quad
           relax($u \rightarrow v$)
\end{algorithm}
\end{minipage}
\hfill%
\hspace{-4cm}
\begin{minipage}[t]{0.38\linewidth}
\begin{algorithm}
\textbf{\underline{isTense}}($u \rightarrow v$): \\ \quad
return $dist[u] + \tilde w(u \rightarrow v) < dist[v]$ \\

\end{algorithm}

\vspace{0.5cm}

\begin{algorithm}
\textbf{\underline{relax}}($u \rightarrow v$): \\ \quad
$dist[v] \leftarrow dist[u] + \tilde w(u \rightarrow v)$ \\ \quad
pred[v] $\leftarrow u$ \\
\end{algorithm}
\end{minipage}

\vspace{0.5cm}

\begin{oneshot}{Observation}
Each vertex will be added once to the queue.
\end{oneshot}

\begin{oneshot}{Corollary}
Each edge will be relaxed at most once.
\end{oneshot}

\begin{oneshot}{Lemma 2.1}
If there is no tense edge in $G$, then for each 
$v: r \rightarrow \ldots \rightarrow \text{pred}(\text{pred}(v)) \rightarrow 
\text{pred}(v) \rightarrow v$ is the holiest path from $r$ to $v$.
\end{oneshot}
\begin{proof}
Let's prove it by induction on $dist[v].length$ distance from the root
$r$.
\begin{oneshot}{Base}
$dist[v].length = 0$, then v = r, so the claim holds trivially.
\end{oneshot}

\begin{oneshot}{Induction Step}
Suppose the claim is true for all vertex $v \in V$ 
such that $dist[v].length < d$ for some d. Consider vertex v such that 
$dist[v].length=d$. By induction hypothesis, all vertices with 
$dist[u].length = d-1$ have holiest path 
correctly updated. By definition, $dist[v] = \min\limits_{u \rightarrow v}\{
dist[u] + \tilde w(u \rightarrow v)\}$, here $dist[u].length = d-1$. 
By Induction hypothesis, $dist[u]$ is not tense and can construct holiest 
path to $u$, so if there is no tense edge in G then 
$dist[v] = \text{min}\{dist[u] + \tilde w(u \rightarrow v)\}$ 
holds.
\end{oneshot}
\end{proof}

\begin{oneshot}{Corollary}
The algorithm will produce holiest tree rooted at r in $O(n + g)$ time.
\end{oneshot}

\subsection{Moving Along an Edge}

We follow Cabello et al's \cite{cabello2013multiple}
method to obtain shortest tree $T_v$ from $T_u$. In our case, we are given holiest tree $T_u$ and would like to obtain $T_v$, holiest tree rooted at $v$, by performing series of pivots.\\

We bisect the given edge $uv$ and insert new source s, which is connected to
both $u$ and $v$. At the start of the process, $s == u$, and s move continuously from $u$ to
v, and when $s == v$, we would have our $T_s = T_v$. \\

\textbf{Initial attempt:}

Let $(1, \vec{h}, \alpha) = w(u \rightarrow v)$. We treat $\lambda$ as a
parameter with satisfying following equations:

\begin{align}
&w_{u}(s \rightarrow u) = (0, \vec{[0]}, 0), \quad
  w_{u}(s \rightarrow v) = (1, \vec{h}, \alpha) \\
&w_{v}(s \rightarrow u) = (1, -\vec{[h]}, -\alpha), \quad
  w_{v}(s \rightarrow v) = (0, \vec{0}, 0)
\end{align}

The natural definition would be as follows, but it does not satisfy our
constraints $(1)-(2)$.
\begin{align}
&w_{\lambda}(s \rightarrow u) = (\lambda, \vec{[0]}, 0)  \\
&w_{\lambda}(s \rightarrow v) = (1 - \lambda, \vec{h}, \alpha)
\end{align}

Therefore, we modify our parametric definition by decreasing distance to u by
$w(u \rightarrow v) = (0, -\vec{[h]}, -\alpha)$. This change allows us to define

\begin{align}
&w_{\lambda}(s \rightarrow u) = (0, -\vec{[h]}, -\alpha) + 
\lambda * (1, \vec{[h]}, \alpha)  \\
&w_{\lambda}(s \rightarrow v) = (1, \vec{[h]}, \alpha) + 
\lambda * (1, \vec{[h]}, \alpha)
\end{align}

Since we decrease distance to $u$ at the start, this could potentially introduce
pivots. Suppose, for instance, $x$ be descendant of $u$ and $y$ be of $v$ in $T_s$,
then:
\begin{align*}
& slack(x \rightarrow y) = dist(x) + w(x \rightarrow y) - dist(y) = \\
& = dist_0(x) + (0, -\vec{[h]}, -\alpha) + w(x \rightarrow y) - dist_0(y) = \\
& = slack_0(x \rightarrow y) + (0, -\vec{[h]}, -\alpha) \Rightarrow \\
& slack(x \rightarrow y) < 0, \text{ if } slack_0(x \rightarrow y) < 
(0,\vec{[h]},\alpha)
\end{align*}

%%%%%%%%%%%%%%%%%%%%%%%%%%%%%%%%%%%%%%%%
\iffalse
\begin{center}
{\color{red}
1. First attempt of not maintain holiest tree at all times.
\\
2. Second attempt of continuously moving by changing homology and 
$\alpha$ terms, also reducing destination distance initially.
\\ 
3. Necessity of resolving homology and $\alpha$ values for an edge 
that source is moving.}
\end{center}
\fi
%%%%%%%%%%%%%%%%%%%%%%%%%%%%%%%%%%%%%%%%

Furthermore, we need to take into account that $u \rightarrow v$ could potentially have non-trivial homology $\vec{[h]}$ and $\alpha$ term, in which case we need to define what it means to split $uv$ to two separate edges $us$ and $sv$, and how to resolve $\vec{[h]}$ and $\alpha$ terms. \\

\textbf{Refined attempt:} \\
To deal with the issue, we reduce distances to u and v as follows:

\begin{align}
& w_{\epsilon}(s \rightarrow u) = ( 0, -\vec{[h]}, -\alpha ) \\ 
& w_{\epsilon}(s \rightarrow v) = ( 1, \vec{[0]}, 0 )
\end{align}

This process can explained in following way: 
\begin{enumerate}
\item When we split $uv$ to 2 edges $(us, sv)$, we assume $\vec{[h]}, \alpha$ terms reside in $sv$ side of $uv$ at the start.
\item By reducing weight of $w(s \rightarrow v) $ by $( 0, -\vec{[h]}, -\alpha )$, we are removing homology and $\alpha$ terms from $sv$ edge.
\item Then we add those values to $su$, which is equivalent to adding  $( 0, -\vec{[h]}, -\alpha )$ to dart $w(s \rightarrow u) $.
\end{enumerate}

Since we reduced distance to all vertices in the graph by equal amount (this is because each vertex is either child of $u$ or $v$ in $T_s$, so reducing distances to $u$ and $v$ is same as reducing distances to all vertices), the process does
not introduce any pivots. We define a parametric weights as follows:

\begin{align}
& w_{\lambda}(s \rightarrow u) = ( 0, -\vec{[h]}, -\alpha ) - \lambda \\
& w_{\lambda}(s \rightarrow v) = ( 1, \vec{[0]}, 0 )  + \lambda 
\end{align}

Every other dart $x \rightarrow y$ has constant parametric weight 
$w_{\lambda}(x \rightarrow y) = w(x \rightarrow y)$.
We then maintain the holiest tree $T_{\lambda}$ rooted at s, with respect
to the weight function $w_{\lambda}$, as $\lambda$ increases continuously from
0 to $( 1, \vec{[0]}, 0 )$. Observe that $T_{\lambda} = T_v$, when $\lambda = w(u \rightarrow v)$. \\

In the following algorithm, \textbf{pred} defines holiest tree rooted at $u$, and 
\textbf{dist} is corresponding distance to each vertex in the graph. \\

% To ensure that introducing $dist_0[s \rightarrow u] = ( 0, -2 * [
% w(u \rightarrow v)], -2 * \alpha(w(u \rightarrow v)) )$ does 
% not introduce any dart with negative slack, we carefully choose fundamental 
% homology cycles to not cross any of the darts of boundary face f. If we assume
% that moving around the face process is counter-clockwise as shown below,
% $\forall u_i -> u_{i + 1} \in f, \alpha(w(u_i \rightarrow u_{i + 1})) <= 0$.
% Therefore, $dist[s \rightarrow u] = ( 0, -2 * [
% w(u \rightarrow v)], -2 * \alpha(w(u \rightarrow v)) )  + \lambda
% = ( 0, [0], -2 * \alpha(w(u \rightarrow v)) )  + \lambda$. Since we
% are adding non-negative distance to u, no slack will be nonnegative by
% construction. 

\begin{center}
\begin{algorithm}
\textbf{\underline{MoveAlongEdge}}($G, u \rightarrow v, dist, pred$): \\ \quad
Add new vertex s \\ \quad
$pred[u], pred[v] \leftarrow s$ \\ \quad
$\lambda \leftarrow 0$ \\ \\ \quad

$w(s \rightarrow u) \leftarrow ( 0, -[w(s \rightarrow u)], 
-\alpha(w(s \rightarrow u)) )$ \\ \quad
AddSubtree$(( 0, -[w(s \rightarrow u)], 
-\alpha(w(s \rightarrow u)) ), u)$ \\ \\ \quad

$w(s \rightarrow v) \leftarrow ( 1, \vec{[0]}, 0 )$ \\ \quad
AddSubtree$(( 0, -[w(s \rightarrow u)], 
-\alpha(w(s \rightarrow u)) ), v)$ \\ \\ \quad

while $\lambda < ( 1, \vec{[0]}, 0 )$: \\ \quad \quad
    \textbf{pivot} $\leftarrow $ FindNextPivot \\ \quad \quad
    If \textbf{pivot} is non NULL \textbf{AND} 
    $(\lambda + slack(\textbf{pivot}) / 2) < ( 1, \vec{[0]}, 0 )$ \\ \qquad \quad
        Pivot(\textbf{pivot}) \\ \qquad \quad
        $\lambda \leftarrow \lambda + slack(\textbf{pivot}) / 2$ \\ \quad \quad
    else \\ \qquad \quad
        $\delta = ( 1, \vec{[0]}, 0 ) - \lambda$ \\ \qquad \quad
        AddSubtree$(\delta, u)$ \\ \qquad \quad
        AddSubtree$(-\delta, v)$ \\ \qquad \quad
        $\lambda \leftarrow \lambda + \delta$ \\
\end{algorithm}
\end{center}
\vspace{1cm}
Below, we show an example of moving along an edge process on a genus $g=2$ grid graph. \\
\begin{minipage}[t]{0.48\linewidth}
\begin{align*}
& \qquad \qquad \qquad \textbf{Primal $T$} \\
& \text{{\color{red} Red} darts are increasing in distance.} \\
& \text{{\color{blue} Blue} vertices are decreasing in distance.} \\
& \text{Pivoting \textbf{IN} dart is denoted with \textbf{i}.} \\
& \text{Pivoting \textbf{OUT} dart is denoted with {\color{electricpurple} o}.} \\
\end{align*}
\end{minipage}
\hfill%
\hspace{-4cm}
\begin{minipage}[t]{0.48\linewidth}
\begin{align*}
& \qquad \qquad \quad \textbf{Dual $(G/T)^*$} \\
& \text{{\color{red} Red} darts connect red vertices in $T$.} \\
& \text{{\color{blue} Blue} darts connect blue vertices in $T$.} \\
& \text{Active darts are denoted with {\color{green} green}} \\
& \text{Pivoting \textbf{IN} dart is denoted with {\color{electricpurple} i}.} \\
& \text{Pivoting \textbf{OUT} dart is denoted with \textbf{o}.}
\end{align*}
\end{minipage}

\begin{figure}
\centering
\begin{subfigure}{.5\textwidth}
  \vspace{-6.0\baselineskip}
  \includegraphics[width=0.8\linewidth,angle=-90]{figures/g2_primal_1.png}\\[9ex]
  \includegraphics[width=0.8\linewidth,angle=-90]{figures/g2_primal_2.png}\\[9ex]
  \includegraphics[width=0.8\linewidth,angle=-90]{figures/g2_primal_3.png}
\end{subfigure}%
\begin{subfigure}{.5\textwidth}
  \vspace{-5.1\baselineskip}
  \includegraphics[width=1.1\linewidth,angle=-8]{figures/g2_dual_1.png}\\[0.5ex]
  \includegraphics[width=1.1\linewidth,angle=-8]{figures/g2_dual_2.png}
  \includegraphics[width=1.1\linewidth,angle=-8]{figures/g2_dual_3.png}
\end{subfigure}
\end{figure}

\newpage
\section{Bounding number of pivots}

\begin{oneshot}{Lemma 4.1}
Let $v_0$ be a first vertex in our given face $f$, and $v_i$ be the source right
after $i^{th}$ pivot. Consider a vertex $y$. We denote holiest path from 
$v_i$ to $y$ as $P_i^{y}$. Then $P_i^{y}$ and $P_j^{y}$ are non-crossing for 
all $i, j$.
\end{oneshot}
\begin{proof}
Suppose there is a crossing and let $z$ be the last crossing point of $P_i^{y}$, $P_j^{y}$. $P_i^{z \rightarrow y}$, $P_j^{z \rightarrow y}$ be respective paths from $z$ to $y$. According to the \textbf{Lemma 2.1}, there must be exactly one holiest path from $z$ to $y$, which contradicts with definition of $P_i^{z \rightarrow y}$, $P_j^{z \rightarrow y}$ being distinct holiest paths from $z$ to $y$.
\end{proof}

We introduce a clocking lemma to prove that each edge is involved in pivoting 
process at most $O(g)$ times.

\begin{oneshot}{Lemma 4.2}
As source vertex $s$ moves around the given face f, any dart $d$ has exactly
$2g$ continuous clock state: 
\begin{align*}
& d \in T \\
& d^{*} \in (G/T)^{*} \\
& rev(d) \in T \\
& rev(d^{*}) \in (G/T)^{*}
\end{align*}
\end{oneshot}
\begin{proof}
{\color{red}???}
\end{proof}


\begin{oneshot}{Theorem 4.1}
Total running time of MSSP is $O(gn)$.
\end{oneshot}
\begin{proof}
Building initial holiest tree takes $O(n + g)$ time. The process of moving around the
face and pivoting takes $O(gn)$ as each dart enters the holiest tree and get replaced by a some edge in holiest tree $O(g)$ times.
\end{proof}

\section{Finding pivot quickly}
Cabello et al.\cite{cabello2013multiple} use a grove data structure, a collection of separate trees with some vertices in original graph are replicated, to quickly find the next pivot. They maintain each tree in a dynamic tree and show that the total complexity of finding the next pivot is $O(g\log{n})$. 
\begin{itemize}
\item What data structure do we maintain in the G*?
Finding shortest path in network can also be understood as a Linear Programming 
problem as follows:
\item How do we find next pivot quickly using above structure?
\end{itemize}

\section{Analysis}
\begin{itemize}
\item Building initial tree
\item Pivoting
\item Number of times each edge is pivoted
\item Overall running time
\end{itemize}

%%%%%%%%%%%%%%%%%%%%%%%%%%%%%%%%%%%%%%%%%%%%
\iffalse
\newpage
\section{Working on examples:}

On genus $g = 2$ surface:
\begin{center}
\includegraphics[angle = -90, scale = 0.7]{figures/genus2.jpg}
\end{center}

\newpage
On genus $g = 3$ surface:
\begin{center}
\includegraphics[angle = -90, scale = 0.7]{figures/genus3.jpg}
\end{center}

\newpage
On genus $g = 4$ surface with initial pivot:
\begin{center}
\includegraphics[angle = -90, scale = 0.7]{figures/genus4_initialPivot.jpg}
\end{center}

\newpage
On genus $g = 4$ surface with initial pivot:
\begin{center}
\includegraphics[angle = -90, scale = 0.7]{figures/genus4_after4Pivots.jpg}
\end{center}

\newpage
On genus $g = 5$ surface with initial pivot:
\begin{center}
\includegraphics[angle = -90, scale = 0.7]{figures/genus4_after5Pivots.jpg}
\end{center}
\fi
%%%%%%%%%%%%%%%%%%%%%%%%%%%%%%%%%%%%%%%%%%%%

\newpage
% Show all the references in citation.bib
\nocite{*}
\printbibliography

\end{document}
